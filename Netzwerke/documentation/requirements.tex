\chapter{Anforderungen}

\textbf{Titel}: Entwurf und erste Implementierungen einer Security Work Bench an der thi \\
\textbf{Betreuer}: Prof. Dr. Ernst Göldner\\
\textbf{Beschreibung}:
Ziel dieses Studentenprojekts im WS 15/16 ist die Planung und Erstellung der ersten Teile einer Se-
curity Work Bench. Dies soll eine Umgebung werden, in der Praktika bzw. Übungen zu den Security
Vorlesungen durchgeführt werden können. Die Security Work Bench könnte auch als Umgebung für
weiterführende Bachelor-/ Master-Arbeiten in diesem Gebiet eingesetzt werden.\\
Für das Projekt im WS 15/16 ist angedacht:

\begin{itemize}
\item Entwicklung eines Grobkonzepts für die Security Work Bench (langfristiges Konzept).
\item Auswahl der Implementierungen für dieses Semester, Organisation und Projektplanung dafür.
\item Umsetzung für die ausgewählten Module (Entwurf, Aufbau und Erprobung, Dokumentation, Erstellen der Versuchsanleitungen)
\item Erste Vorschläge für das Projekt im WS 15/16 (abhängig von der Anzahl der Teilnehmer):
	\begin{itemize}
	\item Entwicklung eines Wettbewerbs zur Computer-/ Netzsicherheit („Capture the Flag“) ähnlich dem iCTF: Die Teilnehmer bekommen zu Beginn des Wettbewerbs Server zugewiesen. Auf diesen sind Programme installiert welche am Laufen gehalten werden müssen. Zunächst gilt es Server und Programme zu analysieren, Schwachstellen zu erkennen und Sicherheitslücken zu schließen. Gleichzeitig sollen genau diese erkannten Schwachstellen dafür ausgenutzt werden, um die Gegner zu attackieren. Für all dies gibt es Punkte. Außerdem erhält man Punkte, wenn eigene Programme trotz Attacken am Laufen gehalten werden.
	\item Angriffe auf moderne Netzwerke: Im Rechnernetze-Labor sind zahlreiche Router, Ethernet Switches, Server und auch eine Firewall vorhanden. Mit diesen Geräten sollen Übungen zur Demonstration klassischer Angriffsszenarien entwickelt werden. Im zweiten Schritt sollen dann auch Maßnahmen zum Schutz untersucht und ggfs. implementiert werden.
	\item Klassische Schwachstellen in den Betriebssystemen: Entwicklung von Übungen, die bekannte Schwachstellen demonstrieren und deren Behebung
	\item Sicherheit von Passwörtern: Entwicklung von Übungen, die bekannte Schwachstellen demonstrieren und deren Behebung
	\end{itemize}
\end{itemize}

\textbf{Ziele}:\newline
\begin{itemize}
\item mit Abschluss dieses Projekts soll eine umfassender Plan für eine Security Work Bench an der thi erarbeitet sein.
\item Erste Übungen, die begleitend zu den einschlägigen Vorlesungen durchgeführt werden können, sind entworfen, erprobt und mit einer adäquaten Dokumentation vorhanden, so dass im nächsten Semester diese Übungen durchgeführt werden können.
\item Optional können noch weitere Übungen skizziert werden, die eine sinnvolle Ergänzung bzw. Weiterentwicklung dieses Projektes sind und deren Realisierung den Rahmen dieses Semesters übersteigt.
\end{itemize}