\chapter{Szenarios}

\section{ARP-Spoofing}

\section{DNS-Spoofing}

\subsection*{Vorraussetzungen}

\begin{itemize}
\item Kali Linux 2.0
\item ARP-Spoofing
\item dnsspoofing
\end{itemize}


\subsection*{Grundlagen}

\subsubsection*{DNS}
Die Addressierung und der anschließende Verbindungsaufbau zu einem Server erfolgt über eine eindeutige IP-Adresse. Damit der Mensch leichter eine Verbindung zu einem Server aufbauen kann,
wurde das DNS (Domain Name System) eingeführt. Dieses verwendet so genannte Domains zur Idenitifzierung von Servern, beispielweise "www.thi.de", da sich diese leichter merken lassen, als
eine IP-Adresse (z.B. 194.94.240.179). DNS ähnelt damit der Funktionsweise eines Telefonbuchs.
Das Domain Name System ist baumförmig aufgebaut, wie nachfolgende Abbildung illustriert: 
%[https://de.wikipedia.org/wiki/Domain_Name_System#/media/File:Dns-raum.svg]


\subsection*{Szenario}
Ein Client (z.B. Windows-Rechner) möchte die Internetseite der Technischen Hochschule Ingolstadt (www.thi.de) aufrufen. Dazu stellt dieser einen DNS-Request an seinen lokalen DNS-Server.
Wenn dieser in seinem Cache keinen Eintrag findet, frägt er - beginnend am Root-DNS-Server - iterativ alle Nameserver nach ihren Einträgen ab, um zum Schluss die IP-Adresse von www.thi.de
zu erhalten.
Abbildungs-Quelle: The Hacker's Handbook: The Strategy Behind Breaking into and Defending Networks - Seite 335

\subsection*{Technisches}
Um einen DNS-Eintrag für eine Domain, beispielsweise www.thi.de, zu manipulieren, kann mittels DNS Cache Poisoning der lokale DNS-Cache des Clients mit falschen Einträgen "vergiftet" werden.
Da bei jeder DNS-Anfrage eine zufällig generierte Transaktions-ID mitgeschickt wird, und eine DNS-Antwort nur akzeptiert wird, wenn diese mit der Anfrage übereinstimmt, muss man als Angreifer diese ermitteln, was sich in einem lokalen Netzwerk mit einem Sniffer sehr einfach realisieren lässt. Alternativ kann auch die Transaktions-ID erraten werden, wofür für die 16-Bit lange Transaktions-ID im Durchschnitt 32.768 Versuche notwendig sind.

\subsection*{Skript}
DNSSpoofing wurde von Dug Song (www.monkey.org/~dugsong/dsniff) entwickelt und veröffentlicht. Mit Unterstützung dieses Tools ist ein Manipulation des DNS-Caches eines Clients im lokalen
Netzwerk sehr leicht durchzuführen. Das Tool ermittelt die verwendeten Transaktions-ID durch Sniffen der ID, sobald der DNS-Server eine Antwort an den Client sendet. Sobald er die ID der
Anfrage ermittelt hat, muss er eine schnellere Antwort an den anfragenden Client versenden, als der eigentliche DNS-Server. Dies geschieht nach mehrfachen Tests und Analyse durch Wireshark
auch regelmäßig. \\

Benutzung von DNS-Spoofing-Skript: \newline
Um dnsspoof einsetzen zu können, muss initial eine hosts-Datei erstellt werden, die die zu manipulierenden Einträge in folgendem Format enthält: \newline
<IP-Adresse>		<Domain>   \newline
<192.168.20.135>	www.thi.de \newline
(Wichtig ist hierbei die Trennung von IP-Adresse und Domainname durch Tab und keinen Leerzeichen!) \\

Anschließend wird \textit{dnsspoof} mit folgenden Parameter aufgerufen: \newline

\begin{description}
	\item[-i] Interface in dem sich lokales Netzwerk befindet
	\item[-f] Hosts-File, absoluter Pfad zu Ort der erstellten hosts-Datei
\end{description}
	



\subsection*{Gegenmaßnahmen}

\subsubsection*{DNSSEC}
Durch DNSSEC kann die Authenzität einer DNS-Antwort verifiziert werden und somit DNS Cache Poisoning vorgebeugt werden. Durch eine asymmetrische Signatur -  ähnlich PGP - kann der Absender
der DNS-Antwort, also der DNS-Server, seine Antworten signieren, indem er mit dem nur ihm zugänglichen privaten Schlüssel den Record unterschreibt. Die Clientseite kann anschließend im 
Gegenzug die Antwort mit dem öffentlichen Schlüssel des DNS-Servers überprüfen, ob die Antwort auch von dem richtigen Server war.

\section{Denial of Service (DoS)}
ToDo: DoS

\section{SSL-Strip}
ToDo: SSl-Strip