\chapter{Zusammenfassung}

Im Projekt \textit{Security-Workbench} vom Wintersemester 2015/2016 wurden durch Sebastian Schuster und Julian Rieder verschiedene (Angriffs-)szenarien auf ISO/OSI-Layer 1-4 entwickelt.\\

Als Ergebnis können folgende Angriffstechniken vollautomatisiert vorgeführt werden:
\begin{itemize}
\item ARP-Spoofing
	\begin{itemize}
		\item Mitlesen von Datenpakete
		\item Manipulation von Inhalten aus Datenpakete
	\end{itemize}
\item DNS-Spoofing
\item SSL-Strip
\item Fake-IPv6-Netz
\item Denial of Service
\end{itemize}


Zur leichteren und schnelleren Demonstration der Angriffstechniken wurde in Python eine Applikation entwickelt, welche alle Szenarios automatisiert ausführt und der Benutzer lediglich wenige notwendige Parameter (z.B. Ziel-IP-Adresse, Domains, Gateway) eingeben muss. Im Hintergrund werden dann alle erforderlichen Programme und Skripte mit der richtigen Konfiguration gestartet.\\

Um die Wartbarkeit dieses Tools zu erhöhen, wurde eine abstrakte Basisklasse definiert, welche die beiden Methodenrümpfe start() und help() enthält. Alle Angriffsszenarien wurden
außerdem in eigene (abgeleitete) Klassen gekapselt. \\

Damit zukünftige Studenten dieses Projekt weiterentwickeln können, wurde großer Wert auf die Dokumentation gelegt. Alle Angriffsszenarien sind nach diesem
Schema aufgebaut:

\begin{itemize}
\item Voraussetzungen
\item Grundlagen
\item Szenario
\item Technisches
\item Erklärung von erforderlichen Tools
\item Benutzung des Python-Skripts
\item Gegenmaßnahmen
\end{itemize}
