\chapter{Aufgabe: Buffer Overflow}

\section{Ziel}
Das Ziel dieser Aufgabe ist es mithilfe eines Buffer Overflows eine Sicherheitsschranke zu umgehen, um ein Lösungswort zu erhalten.


\section{Beschreibung}
Die Aufgabe ist zweigeteilt bei dem ersten teil muss die Datei Buffer1 genutzt werden im zweiten Teil die Datei Buffer2.

Als Hilfestellung ist der Quellcode angegeben jedoch natürlich ohne Lösungswort. 

Nutzen Sie den gdb.

\section{Hilfestellungen}
Nach einiger Zeit werden auf dem Server Hilfestellungen freizuschalten. Diese können durch die nachfolgenden Befehle ausgelesen werden:

\begin{itemize}
\item Hilfestellung 1: Es ist möglich Skriptsprachen wie perl zu verwenden um Zeichen effektiv einzugeben. Zum Beispiel: ./Buffer1 \$(perl -e 'print "'A"'x20" . "'123"';') übergibt dem Programm Buffer1 20 mal das "'A"' Zeichen gefolgt von "'123"'. Es ist auch möglich Hex Ziffern einzugeben "'x41"' ist das äquivalent zu "'A"'
\item Hilfestellung 2: Little Endian bei hex nicht vergessen!
\end{itemize}

\section{Ergebnis}
Hier wird das Lösungswort sowie der Befehl, der zu dem Lösungswort geführt hat eingetragen: 
