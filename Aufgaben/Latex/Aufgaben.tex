\documentclass[
				11pt,				% Schriftgröße
				a4paper,				% Blattgröße - default A4
				DIV=12,				% Satzspiegel
				BCOR=5mm,			% Bindekorrektur
				oneside,				% Zwei seiten
				toc=bibliography		% Literaturverzeichnis inss TOC
					]{scrreprt}

\parskip=3pt							% Abstand nach oben nach Absatz
\parindent=0pt						% Abstand nach rechts nach Absatz

%deutsches Schriftbild
\usepackage[T1]{fontenc} 			% Schriftcodierung
\usepackage[utf8]{inputenc}	 		% Eingabecodierung
\usepackage[ngerman]{babel} 			% Deutschsprachige Textelemente
\usepackage{graphicx}
\usepackage{float}

% Random-Text
\usepackage{blindtext}

%Linking
\usepackage{hyperref}
\hypersetup
{
colorlinks=true,
linkcolor=black,
citecolor=black,
filecolor=black,
urlcolor=black,
}

%Formelunterstützung
\usepackage{amsmath}
\usepackage{amssymb}

% Tabelle
\usepackage{booktabs}
\usepackage{longtable}

% Literatur
%\usepackage[babel, german=quotes]{csquotes}
%\usepackage[style=authoryear-ibid, backend=biber, bibencoding=utf8]{biblatex}

% Abkürzungsverzeichnis
\usepackage{acronym}

% Code
\usepackage{listings}
\usepackage{color}

 \definecolor{middlegray}{rgb}{0.5,0.5,0.5}
 \definecolor{lightgray}{rgb}{0.8,0.8,0.8}
 \definecolor{orange}{rgb}{0.8,0.3,0.3}
 \definecolor{yac}{rgb}{0.6,0.6,0.1}
 \definecolor{codeblue}{rgb}{0.0,0.0,1.0}
 \definecolor{codegrey}{rgb}{0.9, 0.9, 0.9}


 \lstset{
   backgroundcolor=\color{codegrey},
   basicstyle=\scriptsize\ttfamily,
   keywordstyle=\bfseries\ttfamily\color{orange},
   stringstyle=\color{codeblue}\ttfamily,
   commentstyle=\color{middlegray}\ttfamily,
   emph={square}, 
   emphstyle=\color{blue}\texttt,
   emph={[2]root,base},
   emphstyle={[2]\color{yac}\texttt},
   showstringspaces=false,
   flexiblecolumns=false,
   tabsize=2,
   numbers=left,
   numberstyle=\tiny,
   numberblanklines=false,
   stepnumber=1,
   numbersep=10pt,
   xleftmargin=15pt
 }


% Kopf- / Fußzeile
\usepackage{fancyhdr}

\pagestyle{headings}

\usepackage{tocbibind}

%\addbibresource{refBib.bib}

\begin{document}

\chapter{Aufgabe: Command Injection}

\section{Ziel}
Das Ziel dieser Aufgabe ist es eine ID vom Server auszulesen.
Diese ID ist für jeden Nutzer individuell und muss am Ende dieses Aufgabenblattes eingetragen werden.

\section{Beschreibung}
Sie werden den Service Fahrgestellnummer mithilfe einer Command Injection angreifen um eine Nummer aus der Datei HIER DATEI FÜR DIE IDS auszulesen.

Um auf den Service zuzugreifen starten Sie ein Terminal.
Dort können Sie sich mit NetCat auf den Zielserver einloggen.
Dazu geben Sie folgenden Befehl ein:

\begin{lstlisting}[language=bash]
 nc 1.0.0.0 1337
\end{lstlisting}

Anschließend können Sie bei dem Service eine Fahrgestellnummer eingeben um den NOx Ausstoß zu testen. \\
Dieser Service soll nun von Ihnen angegriffen werden um die ID aus HIER WIEDER auszulesen. 

\section{Hilfestellungen}
Nach einiger Zeit werden auf dem Server Hilfestellungen freizuschalten. Diese können durch die nachfolgenden Befehle ausgelesen werden:

\begin{itemize}
\item Hilfestellung 1: nach 30 min
\item Hilfestellung 2: nach 45 min
\item Hilfestellung 2: nach 60 min
\end{itemize}

\section{Ergebnis}
Hier wird die erhaltene einzigartige ID eingetragen:
\\
\\

\underline{\hspace{1cm}} 
\underline{\hspace{1cm}} 
\underline{\hspace{1cm}} 
\underline{\hspace{1cm}} 
\underline{\hspace{1cm}} 
\underline{\hspace{1cm}} 
\underline{\hspace{1cm}} 
\underline{\hspace{1cm}} 
\underline{\hspace{1cm}} 
\underline{\hspace{1cm}} 

\chapter{Aufgabe: TITEL}

\section{Ziel}
Das Ziel dieser Aufgabe ist es eine ID vom Server auszulesen.
Diese ID ist für jeden Nutzer individuell und muss am Ende dieser Aufgabenblattes eingetragen werden.

\section{Beschreibung}

\section{Hilfestellungen}
Nach einiger Zeit werden auf dem Server Hilfestellungen freizuschalten. Diese können durch die nachfolgenden Befehle ausgelesen werden:

\begin{itemize}
\item Hilfestellung 1: nach 30 min
\item Hilfestellung 2: nach 45 min
\item Hilfestellung 2: nach 60 min
\end{itemize}

\section{Ergebnis}
Hier wird die erhaltene einzigartige ID eingetragen:
\\
\\

\underline{\hspace{1cm}} 
\underline{\hspace{1cm}} 
\underline{\hspace{1cm}} 
\underline{\hspace{1cm}} 
\underline{\hspace{1cm}} 
\underline{\hspace{1cm}} 
\underline{\hspace{1cm}} 
\underline{\hspace{1cm}} 
\underline{\hspace{1cm}} 
\underline{\hspace{1cm}} 

\chapter{Aufgabe: TITEL}

\section{Ziel}
Das Ziel dieser Aufgabe ist es eine ID vom Server auszulesen.
Diese ID ist für jeden Nutzer individuell und muss am Ende dieser Aufgabenblattes eingetragen werden.

\section{Beschreibung}

\section{Hilfestellungen}
Nach einiger Zeit werden auf dem Server Hilfestellungen freizuschalten. Diese können durch die nachfolgenden Befehle ausgelesen werden:

\begin{itemize}
\item Hilfestellung 1: nach 30 min
\item Hilfestellung 2: nach 45 min
\item Hilfestellung 2: nach 60 min
\end{itemize}

\section{Ergebnis}
Hier wird die erhaltene einzigartige ID eingetragen:
\\
\\

\underline{\hspace{1cm}} 
\underline{\hspace{1cm}} 
\underline{\hspace{1cm}} 
\underline{\hspace{1cm}} 
\underline{\hspace{1cm}} 
\underline{\hspace{1cm}} 
\underline{\hspace{1cm}} 
\underline{\hspace{1cm}} 
\underline{\hspace{1cm}} 
\underline{\hspace{1cm}} 

\chapter{Aufgabe: TITEL}

\section{Ziel}
Das Ziel dieser Aufgabe ist es eine ID vom Server auszulesen.
Diese ID ist für jeden Nutzer individuell und muss am Ende dieser Aufgabenblattes eingetragen werden.

\section{Beschreibung}

\section{Hilfestellungen}
Nach einiger Zeit werden auf dem Server Hilfestellungen freizuschalten. Diese können durch die nachfolgenden Befehle ausgelesen werden:

\begin{itemize}
\item Hilfestellung 1: nach 30 min
\item Hilfestellung 2: nach 45 min
\item Hilfestellung 2: nach 60 min
\end{itemize}

\section{Ergebnis}
Hier wird die erhaltene einzigartige ID eingetragen:
\\
\\

\underline{\hspace{1cm}} 
\underline{\hspace{1cm}} 
\underline{\hspace{1cm}} 
\underline{\hspace{1cm}} 
\underline{\hspace{1cm}} 
\underline{\hspace{1cm}} 
\underline{\hspace{1cm}} 
\underline{\hspace{1cm}} 
\underline{\hspace{1cm}} 
\underline{\hspace{1cm}} 

\chapter{Aufgabe: TITEL}

\section{Ziel}
Das Ziel dieser Aufgabe ist es eine ID vom Server auszulesen.
Diese ID ist für jeden Nutzer individuell und muss am Ende dieser Aufgabenblattes eingetragen werden.

\section{Beschreibung}

\section{Hilfestellungen}
Nach einiger Zeit werden auf dem Server Hilfestellungen freizuschalten. Diese können durch die nachfolgenden Befehle ausgelesen werden:

\begin{itemize}
\item Hilfestellung 1: nach 30 min
\item Hilfestellung 2: nach 45 min
\item Hilfestellung 2: nach 60 min
\end{itemize}

\section{Ergebnis}
Hier wird die erhaltene einzigartige ID eingetragen:
\\
\\

\underline{\hspace{1cm}} 
\underline{\hspace{1cm}} 
\underline{\hspace{1cm}} 
\underline{\hspace{1cm}} 
\underline{\hspace{1cm}} 
\underline{\hspace{1cm}} 
\underline{\hspace{1cm}} 
\underline{\hspace{1cm}} 
\underline{\hspace{1cm}} 
\underline{\hspace{1cm}} 

\chapter{Aufgabe: TITEL}

\section{Ziel}
Das Ziel dieser Aufgabe ist es eine ID vom Server auszulesen.
Diese ID ist für jeden Nutzer individuell und muss am Ende dieser Aufgabenblattes eingetragen werden.

\section{Beschreibung}

\section{Hilfestellungen}
Nach einiger Zeit werden auf dem Server Hilfestellungen freizuschalten. Diese können durch die nachfolgenden Befehle ausgelesen werden:

\begin{itemize}
\item Hilfestellung 1: nach 30 min
\item Hilfestellung 2: nach 45 min
\item Hilfestellung 2: nach 60 min
\end{itemize}

\section{Ergebnis}
Hier wird die erhaltene einzigartige ID eingetragen:
\\
\\

\underline{\hspace{1cm}} 
\underline{\hspace{1cm}} 
\underline{\hspace{1cm}} 
\underline{\hspace{1cm}} 
\underline{\hspace{1cm}} 
\underline{\hspace{1cm}} 
\underline{\hspace{1cm}} 
\underline{\hspace{1cm}} 
\underline{\hspace{1cm}} 
\underline{\hspace{1cm}} 




\end{document}

