\chapter{Dungeon}


\section{Services: Dungeon}
The Dungeon service was a consoleservice which presented a litte game. The player needs to run through the dungeon to finally kill a dragon in the end. To do this, two items are needed. Firstly a golden sabre and second a magic shield. If either of the items is missing the player will die to the dragon.
To get the sabre the player needs to find a secret room and in there has to get the name of an airplane picture right. If he does he will get the sabre. 
Now the player will get in a room with a dwarf. He says if you get the number i guess you will get my shield. But the number he has is random generated and nearly impossible to get right. Also if you get the number right he will tell you that he gives you the shield but the variable HAVE\_SHIELD will not change. To achive the shield the player has to go to a room with a gnome which asks the player's name and then says Hi! playername. In this function there is an adress returned with the variable HAVE\_SHIELD. So the player need to type in anycharacter than \%n so that the adress will be overwritten with 0x1.
If the player encounters the dragon with both items he can slay the dragon with the command "'kill dragon"'. After that he will be asked what his name is. Here comes the tricky part. With a Bufferoverflow it is possible to alter the return adress so that the function will jump to the secret treasure room where the flag can be optained. To do this the player have to type 88 random characters followed by 0xDEADBEEF in ASCII than 8 random characters again and then the adress to the storage room. The 0xDEADBEEF is needed because a value is checked for change. If you overwrite it with the bufferoverflow the service will disconnect the player. Both 0xDEADBEEF and the return adress have to be turned because little Endian is used. 


%Der Dungeon Service war ein Consolenservice der ein Spiel darstellt. der Spieler muss dabei durch ein Dungeon laufen und am Ende einen Drachen töten. Um dies zu bewerkstelligen werden 2 Gegenstände gebraucht. Ein goldener Säbel und ein magisches Schild. 
%Um den Säbel zu bekommen muss zuerst ein geheimer Raum gefunden werden in dem der Namen eines Flugzeuges anhand eines Abbilds erraten werden muss. Wird richtig geraten erhält der Spieler den Säbel. 
%Des weiteren gibt es einen Raum, bei dem ein Zwerg sein Schild abgibt, wenn du seine Nummer errätst. Dies ist jedoch fast unmöglich da die Nummer zufällig bestimmt wird. Des weiteren bekommt man den Schild ebenfalls nicht wenn man die Nummer errät. 
%Um diesen nun zu bekommen muss in einem anderen Raum bei einem Gnom der nach den Spielernamen fragt den Speicher überschreiben. Die Funktion übergibt die Adresse der Variable HAVE\_SHIELD und wenn man ein beliebiges Zeichen und dann \%n eingibt, wird auf der Adresse HAVE\_SCHIELD eine 1 eingetragen. 
%Wenn der Spieler nun beim Drachen ist, stirbt er nicht wie sonst, sondern kann den Drachen mit dem command  "'kill dragon"' töten. 
%Im Anschluss daran wird er Spieler erneut nach seinem Namen gefragt. Nun kann man mithilfe eines Bufferoverflows die Rücksprungadresse der Funktion so ändern, dass der Spieler im Secret Storage Room landet. Dort wird die Flagge ausgegeben, wenn die ID angegeben wird. Eine Hürde beim Bufferoverflow ist jedoch, dass auf eine Variable geprüft wird und wenn die überschrieben wird, wird der Nutzer vom Service getrennt. Um dies zu verhindern muss genau ab der 88. Stelle 0xDEADBEEF in ASCII mitübergeben werden. Danach noch 4 Zufallszeichen und dann die Rücksprungadressse. 0xDEADBEEF sowie die Rücksprungadresse müssen umgedreht werden, das litte Endian verwendet wird.