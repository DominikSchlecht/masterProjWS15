\chapter{Wireless Security}

\section{Einleitung}

\section{Vorbereitungen}

Voraussetzungen für die weiteren Übungen:
- Alfa USB WLAN-Adapter
- Workstation/Notebook
- Virtual Box mit Kali Linux Image (+ Extension Pack)
- Grundlegende Kenntnisse mit Linux

Konfiguration des Alfa Adapters in Kombination mit Virtual Box

NICHT NOTWENDIG1. Anschluss des Adapters über das beigelegte Y-Kabel an den Host
NICHT NOTWENDIG2. Virtual Box die Kali Maschine auswählen -> Rechtsklick: Ändern -> USB auswählen -> USB-2.0-Controller aktivieren ->	USB-Filter hinzufügen
-> Ralink WLAN auswählen (falls nicht vorhanden im GeräteManager nach dem WLAN Adapter suchen) -> Mit OK bestätigen
	Zusätzlich einen neuen Filter anlegen und Hersteller- bzw. Produkt-ID aus dem ersten Filter übernehmen. Die anderen Felder können unausgefüllt werden
NICHT NOTWENDIG3. WLAN Adapter ausstecken\\

X. Kali Linux in Virtual Box starten (user: root, passwort: toor)\\
X. Sobald die VM hochgefahren ist, den Adapter einstecken\\
X. USB Icon im Fenster der Maschine sollte rot/grün blinken (evtl screenshot)\\

\section{WEP}

\section{WPA/WPA2}
%Teilweise aus:
%http://www.elektronik-kompendium.de/sites/raspberry-pi/2008101.htm

WPA bzw. WPA2 (WiFi Protected Access) ist eine Kombination aus Authentifizierung und Verschlüsselung, um ein WLAN sicher zu betreiben. Die Authentifizierung erfolgt in der Regel mit einem Passwort, um den Zugriff durch unberechtigte Personen zu verhindern. Möchte ein Angreifer nun in das Netzwerk eindringen, muss er dieses Passwort herausfinden.\\


Grundsätzlich gibt es beim Hacken keine Unterschiede zwischen WPA- und WPA2-gesicherte WLANs. Die Authentifizierungsmethode ist im Prinzip identisch. Der Unterschied liegt im Verschlüsselungsverfahren, welche für die typischen Hacking-Methoden auf WPA-gesicherte WLANs nicht relevant ist.\\ Grund dafür ist, dass WPA2 derzeitig noch als nicht zu knackendes Verschlüsselungsverfahren
gilt und daher ein Angriff auf die Verschlüsselung vergebene Mühe wäre. \\

Der typische Angriff gegen ein WPA/WPA2-gesichertes WLAN läuft über reines Bruteforcing oder einer sogenannten Wörterbuch-Attacke (engl. dictionary-attack). Bei ersterem werden einfach alle Kombinationen bestehend aus Buchstaben, Ziffern und Sonderzeichen, oder nur einem Ausschnitt davon, bis zur gewünschten Länge getestet. Je nach Länge und Komplexität des Passworts kann sich dieser Vorgang über viele Stunden, bis zu Tagen und sogar mehreren Jahren hinziehen. Häufig wird bei einer Bruteforce-Attacke zuvor eine Wordlist, wie bei einem Dictionary-Angriff, mit allen zu testenden Kombinationen erstellt. Bei einem Wörterbuch-Angriff wird somit durch die Passwortkandidaten in einer riesigen Wordlist iteriert und mit dem herauszufindenden Passwort abgeglichen. %wie abgeglichen? verhasht ,bzw. verschlüsselt? 
Stimmen beide überein, wurde das Passwort gefunden. Diese Wörterlisten können entweder selber generiert werden oder sind auch im Internet zu finden. Wie wir später noch sehen werden, gibt es auch hybride Ansätze, die beide Angriffsarten verknüpfen.\\


Ein WPA-Handshake findet zwischen Access Point und WLAN-Client statt, wenn der WLAN-Client sich mit dem WLAN verbinden will. Dieser WPA-Handshake muss aufgezeichnet werden. Anschließend wird bei einem Wörterbuch-Angriff mit Hilfe der Wordlist das WLAN-Passwort erraten. Ein erfolgreicher Angriff steht und fällt mit einer guten Wordlist, in der das WLAN-Passwort enthalten sein muss. Darin besteht die eigentliche Schwierigkeit bei einem WPA/WPA2-WLAN-Hack.\\


Grober Ablauf eines WPA-/WPA2-Hacks:

\begin{enumerate}
\item Wordlist erstellen oder besorgen 
\item Grundzustand herstellen und Monitor Mode einschalten
\item WLAN mit WPA/WPA2 identifizieren (Information Gathering) 
\item Datenverkehr mit Airodump-ng aufzeichnen
\item Deauthentication-Attacke mit Aireplay-ng (optional)
\item WPA-Passwort mit Hilfe der Wordlist herausfinden
\end{enumerate}

%more space

\textbf{\Large{Cracking des WPA Keys}}\\ %(evtl mit Screenshots oder weiter ausformulieren)

{\Large 1. Check des WLAN Adapter}\\

Zuerst muss geprüft werden, ob der eingesteckte USB WLAN-Adapter erkannt wird und somit einsatzbereit ist. Dazu das Terminal öffnen in Kali Linux öffnen und folgenden Befehl eingeben. 

$$iwconfig$$\\

Der Adapter sollte als Interface, meist WLAN0 oder WLAN1, angezeigt werden\\
Im Folgenden muss bei allen Befehlen die Interface Bezeichnung mit der hier angezeigten ersetzt werden, da sie sich von Rechner zu Rechner unterscheiden kann.\\


{\Large 2. MAC-Spoofing}\\

Im Sinne von Wireless Security sollte man sich immer im Klaren sein, dass ein Angreifer immer in der Lage ist seine MAC-Adresse zu verändern. Dieser Vorgang wird auch Spoofing genannt.

Die MAC-Adresse ist eine herstellerspezifische Kennung, die fest einem Netzwerkgerät zugeordnet ist. Jede Adresse ist eindeutig. Findet man die MAC-Adresse eines Angreifers heraus, kann mit Hilfe dieser Identifikationskennung festgestellt werden, welchen Typ von Antenne er verwendet. Diese Erkenntnis kann helfen einen Angreifer zu identifizieren.
Verwendet ein Angreifer nun eine gefälschte MAC-Adresse können keine Rückschlüsse auf seine Identität gezogen werden, da überall nur seine Fake-Adresse angezeigt wird.\\

Zuerst muss dafür das WLAN Interface deaktiviert werden. Danach kann mit dem Kommando \textit{macchanger} die Adresse geändert werden.\\
\begin{equation*}
\begin{split}
ifconfig~wlanX~down\\
macchanger~\text{-}r~wlanX
\end{split}
\end{equation*}

\textit{X = NUM für das interface}\\

Beim Bestätigen des Befehls mit Enter, wird die eigene MAC-Adresse in eine zufällige generierte MAC-Adresse geändert und auf der Konsole angezeigt. Anschließend kann das  
Interface mit dem Befehl\\ 
$$ifconfig~wlanX~up$$

\textit{X = NUM für das interface}\\

wieder aktiv gesetzt werden.\\


Mit dem Befehl\\ 
$$ifconfig~wlanX$$\\
kann überprüft werden, ob die gespoofte MAC-Adresse auch aktiv ist.\\

{\Large 3. Das Interface in den Monitor Mode versetzen}\\

Damit mit dem WLAN Adapter Pakete aufgezeichnet werden können, muss sich der Adapter im Monitor Mode, oder auch Packet Injection Mode genannt, befinden. Dies wird mit folgendem Befehl erreicht.

$$airmon\text{-}ng~start~wlanX~$$

\textit{X = your number from iwconfig}\\

Mit dem Befehl\\ 

$$airmon\text{-}ng~check~kill$$

\textit{X = NUM für das interface}\\


werden alle andere Prozesse beendet, die auch auf den Netzwerkadapter zugreifen können. So können Konflikte beim Zugriff auf die Ressource vermieden werden.\\

{\Large 4. Aufzeichnen der WLAN Pakete mit airodump}\\
	
Im nächsten Schritt werden die WLAN Pakete aus der Umgebung aufgezeichnet. Damit möchte man einen Handshake zwischen dem zu hackenden Access Point und einem Client aufzeichnen. Anhand dessen kann anschließend das Passwort herausgefunden werden.\\

Mit dem folgenden Befehl können wir in den Aufzeichnungsmodus umschalten.
	
	$$airodump\text{-}ng~\text{-}b~a~wlanXmon$$\\
	 
	\textit{X = NUM für das interface}\\
	\textit{-b a = Scan im 5GHz Band}\\	
	
Falls wir im im 5GHz Bereich scannen möchten muss der Parameter \textit{-b a} mitgegeben werden. Falls nicht, kann der Parameter einfach weggelassen werden.\\	
Sollten keine Daten aufgezeichnet werden, dann den Adapter mehrmals aus- und wieder einstecken. 
Nach einem Reconnect muss der Adapter natürlich wieder in den Monitoring Modus versetzt werden.
	
Hat alles soweit geklappt, sollten alle erreichbaren SSIDS mit ihren jeweiligen Sendern angezeigt werden.\\ 
	
Als nächstes sollte die MAC-Adresse und der verwendete Kanal des zu hackenden APs notiert.
Anschließend kann durch einen neuen airodump-ng Durchlauf mit der MAC und dem Kanal als Parameter (nähere Infos unter \(man~airodump-ng\) abrufbar) der Scan
	eingeschränkt werden. Zusätzlich kann auch der Name der Ausgabedatei festgelegt werden. 
Der Befehl sieht dann in etwa wie nachfolgend aus.
	$$airodump\text{-}ng~\text{-}c~Kanal~\text{-}b~a~\text{-}\text{-}bssid~MAC\text{-}AP~\text{-}showack~\text{-}w~Filename~wlanXmon$$
	
		\textit{X = NUM für das interface}\\
		\textit{Kanal = der Kanal auf dem gelauscht werden soll}\\
		\textit{MAC-AP = die MAC-Adresse des Access Points}\\
		\textit{Filename = in die zu schreibende Datei}\\

	Verbindet sich nun ein Client auf den AP, so kann der 4-way-handshake mitgelesen werden, was auch in der Konsole, in der rechten oberen Ecke, angezeigt wird.
	Hat dies funktioniert, ist der erste Schritt für das Hacken des Passworts abgeschlossen.\\

{\Large 5. Cracken des Passworts}\\
		
Ab hier werden verschieden Tools und Angriffsarten für das Cracken des Keys vorgestellt.\\	

 \textbf{Dictionary Attack mit aircrack}\\

Dazu wird ein Dictionary File mit allen Passwörtern benötigt, die auf Übereinstimmung mit dem PSK gecheckt werden sollen. Auf dem Image sollt bereits eine Dictionary Datei im Home Verzeichnis vorhanden sein.

Mit folgendem Befehl kann der Dictionary-Angriff gestartet werden. 

$$aircrack\text{-}ng~\text{-}w~dict.file~\text{-}b~MAC\text{-}AP~File.cap$$

\textit{dict.file = Pfad zu dem Dictionary}\\ 
\textit{MAC-AP = Die MAC-Adresse des APs}\\ 
\textit{File.cap = Pfad zu dem cap file}\\

\textbf{Brutefore Angriff mit aircrack und crunch}

\begin{equation*}
\begin{split}
crunch~8~12~abcdefghijklmnopqrstuvwxyzABCDEFGHIJKLMNOPQRSTUVWXYZ \\
|~aircrack-ng~–bssid~00:11:22:33:44:55~-w-~hack-wifi-01.cap
\end{split}
\end{equation*}
 
\textit{8 12 = die zu testende Passwortlängen, hier von Länge 8 bis 12}\\
\textit{abcde.. = die zu testenden Zeichen}\\



\textbf{Attacken mit hashcat}\\

Bei hashcat handelt es sich wohl um den derzeit schnellsten Passwortcracker auf dem Markt. Wir verwenden es als Alternative zu crunch.\\

Convert the .cap file in a hccap file\\

$$aircrack-ng~Filename.cap~-J~newFilename$$

\textit{Filename.cap = Pfad bzw. Name des alten .cap files}\\
\textit{Pfad bzw. Name des neuen .hccap file}\\

%+ Brutefore/Dictionary/Rule-Based Attack with hashcat möglich

Mit hashcat --help kann eine Hilfeseite aufgerufen werden in welcher der Befehl, die Parameter und die Verwendung
genauer erläutert werden. Falls Probleme auftreten oder detailliertere Einstellungen vorgenommen werden sollen, kann 
die Hilfeseite die erste Anlaufstelle sein.\\

\textbf{Dictionary Attack mit hashcat}

$$hashcat~-m~2500~capture.hccap~dict.txt$$

Anschließend nutzt hashcat das Dictionary um das Passwort zu finden.
Mit Enter kann der aktuelle Status des Vorgangs abgefragt werden.\\ 

\textbf{Bruteforce Attack mit hashcat}

$$hashcat~-m~2500~-a3~capture.hccap~?d?d?d?d?d?d?d?d~(?d = 0-9)$$

Bei der Bruteforce Attacke werden alle Kombinationen von Buchstaben bis zu einer bestimmten Länge durchgetestet.
Als letzter Parameter kann eine Art Maske angegeben werden, mit welcher die Länge und die zu testenden Ziffern, Buchstaben
und Zeichen festgelegt werden. Im Beispiel werden alle neun stelligen Zahlenkombinationen von hashcat durchprobiert.\\ 

\textbf{rule-based Attack mit hashcat}

$$hashcat~-m~2500~-r~/usr/share/hashcat/rules/best64.rule~capture.hccap~ dict.txt$$

rule-based attacks gehören zu den komplizierteren Angriffsarten. Dabei wird ein nomaler Dictionary-Angriff gefahren, aber mit rules erweitert. 
Die rules, zu deutsch Regeln, sind wie eine Art Programmiersprache
für die Generierung von Passwörtern. Es gibt Funktionen mit denen Passwortkandidaten bearbeitet, mit anderen Wörtern verknüpft oder bestimmte Kombinationen 
übersprungen werden können. Regeln zu schreiben kann sehr aufwendig sein und erfordert viel Wissen über Passwörter. Daher kann für die ersten Versuche auch 
die best64.rule Regel verwendet werden, die standardmäßig bei hashcat dabei ist.\\


DoS Attack mit MDK3

Erst den Adapter wieder in den Monitoring Mode versetzen.

airmon-ng start wlanX

disconnect bei verwendung von mdk3

\section{WPS}

\section{Sicherungsmaßnahmen}


